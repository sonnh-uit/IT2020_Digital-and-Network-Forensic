\chapter{Phân tích ẩn video (Video stegananalysis)}
Trong khi steganography là quá trình che giấu thông tin bí mật, steganalysis là quá trình phá vỡ thuật toán steganography để phát hiện và khám phá thông tin bí mật được nhúng. Steganalysis được phân loại thành Steganalysis tích cực và Steganalysis thụ động. Phân tích ẩn thụ động chỉ phát hiện sự hiện diện của thông tin bí mật, nhưng phân tích ẩn tích cực phát hiện và giải mã/sửa đổi thông tin bí mật ẩn.

Steganalysis rất quan trọng vì hai lý do: 
\begin{itemize}
    \item Cho kỹ thuật ghi ảnh ẩn có thể đảo ngược trong đó phân tích dữ liệu được yêu cầu ở đầu nhận để trích xuất thông tin bí mật được nhúng.
    \item Để ngăn chặn việc chuyển giao thông tin bất hợp pháp.
\end{itemize}
 
 Có rất nhiều công cụ steganography dễ dàng có sẵn trực tuyến, điều này làm cho nhu cầu phân tích trở nên cấp thiết. Ngay cả một người bình thường cũng có thể dễ dàng truy cập các công cụ ghi mật mã và sử dụng chúng để gửi thông tin bí mật và bị cấm mà không gây bất kỳ nghi ngờ nào cho các quan chức chính phủ.

 Nhu cầu phân tích steganalysis là rất lớn, tuy nhiên, phân tích steganalysis không hề dễ dàng. Với việc lựa chọn phương tiện che phủ phù hợp, ngay cả công cụ phân tích ẩn tốt nhất cũng không thể phá vỡ kỹ thuật ghi ẩn. Vì không có bất kỳ thông tin chi tiết nào về phương tiện truyền thông bìa và nếu không có kiến thức đó, việc phá vỡ bản ghi mật là rất khó. Đặc biệt với video là phương tiện truyền thông bìa, việc phân tích dữ liệu có thể là một thách thức vì khó có thể tương quan giữa các video và các đặc trưng. Các phương pháp phân tích mật được phát triển theo ký hiệu rằng các đặc điểm và đặc trưng của phương tiện che phủ được sửa đổi khi thông tin bí mật ăn sâu vào. Nhiều phương pháp phân tích dữ liệu hoạt động bằng cách so sánh các đặc trưng và đặc điểm khác giữa đối tượng stego và đối tượng che phủ.

 Một phân tích kỹ lưỡng về phương pháp ghi ảnh ẩn hiện có và sự tiến bộ của nó là cần thiết để xây dựng công cụ phân tích ảnh. Một phương pháp phân tích ẩn tốt sẽ có thể phá vỡ các cuộc tấn công ẩn khác nhau và một số phương pháp phân tích ẩn hiện có. Steganalysis được chia thành hai loại, cụ thể là Steganalysis cụ thể và Steganalysis phổ quát. Các phương pháp phân tích ẩn cụ thể được phát triển để xử lý một loại phương pháp ẩn giấu cụ thể. Ví dụ, steganography đảo ngược là một phương pháp stegananalysis cụ thể, vì phương pháp này chỉ có thể phá vỡ một phương pháp steganography cụ thể và có thể không hoạt động hiệu quả với các phương pháp steganography khác. Mặt khác, các phương pháp phân tích ẩn phổ biến nhằm mục đích phá vỡ tất cả các phương pháp lưu trữ. Các phương pháp phân tích ẩn cụ thể có thể thực hiện được, trong khi đó, các phương pháp phân tích ẩn phổ quát lại khó thực hiện. Sự phát triển của các phương pháp phân tích ẩn phổ quát nên theo hướng của các công việc trong tương lai. Nỗ lực phát triển một phần mềm duy nhất có khả năng phá vỡ bất kỳ loại steganography nên được đầu tư.

 \section{Kĩ thuật phân tích ẩn}
 Các phương pháp phân tích ẩn được phân loại thành ba loại chính:
 \begin{itemize}
     \item Phân tích ẩn dựa trên chữ ký
     \item Phân tích ẩn thống kê
     \item Phân tích ẩn dựa trên đặc trưng
 \end{itemize}
 
 Phân tích ẩn chữ ký, như tên cho thấy, sử dụng chữ ký (signature) do phương pháp nhúng để lại để phát hiện sự hiện diện của thông tin bí mật. Phân tích ẩn thống kê sử dụng các phương pháp thống kê và công thức toán học để phát hiện và khám phá thông tin bí mật. Các phương pháp phân tích ẩn dựa trên đặc trưng trích xuất các đặc trưng từ video bìa và video ẩn để điều tra sự hiện diện, từ đó, phát hiện ra thông tin bí mật.

 \subsection{Phân tích ẩn dựa trên chữ ký}
 Các phương pháp phân tích chữ ký được chia thành tấn công trực quan và tấn công cấu trúc. Tính năng đầu tiên và quan trọng nhất được mong đợi từ kỹ thuật ghi video là không thể nhận thấy. Sự biến dạng do phương pháp ghi video gây ra phải ở mức tối thiểu, nếu không, dấu vết của thông báo ẩn có thể hiển thị đối với Hệ thống thị giác của con người (Human Visual System - HVS). Các cuộc tấn công thị giác (Visual
attacks) là kỹ thuật phân tích dữ liệu đơn giản có thể phá vỡ kỹ thuật ẩn ảnh bằng cách sử dụng HVS. Khung ẩn và khung bìa được so sánh cạnh nhau bằng mắt thường để kiểm tra bất kỳ thay đổi có thể nhìn thấy nào. Mặc dù các cuộc tấn công bằng hình ảnh rất dễ thực hiện nhưng chúng không đáng tin cậy. Không chỉ độ tin cậy mà cả tính tự động hóa và yêu cầu chuyên gia thực hiện kiểm tra là những nhược điểm khác của việc sử dụng tấn công trực quan.

Các đặc điểm và đặc trưng của video bìa thay đổi sau khi nhúng thông tin bí mật. Một số đặc điểm, đặc trưng và các thành phần cấu trúc khác của video ẩn được tính đến để phát hiện sự hiện diện của thông tin bí mật. Kiểu tấn công phân tích này được gọi là tấn công cấu trúc. Một ví dụ cụ thể có thể kể đến là việc so sánh kích thước tệp. Sau khi nhúng, kích thước tệp của video bìa dễ bị thay đổi. Tương tự như các cuộc tấn công trực quan, các cuộc tấn công cấu trúc không đáng tin cậy và cần có các chuyên gia trong lĩnh vực này. Không phải tất cả các video bị giả mạo đều sẽ thay đổi cấu trúc và có thể thoát khỏi các phương thức tấn công cấu trúc.

\subsection{Phân tích ẩn thống kê}
Phân tích ẩn thống kê sử dụng các giá trị của pixel hình ảnh và phân tích chúng để phát hiện nội dung bí mật. Phân tích ẩn thống kê ưu việt hơn so với phân tích ẩn dựa trên chữ ký. Các phương pháp thống kê sử dụng kiến thức về các giá trị pixel của ảnh và các mô hình toán học để phát hiện và khôi phục thông tin bí mật. Phân tích ẩn thống kê có thể được nhóm thành phân tích Biểu đồ, Tấn công Chi-Square, Phân tích ẩn RS, nhúng LSB, phân tích cặp pixel khớp LSB, phân tích mặt phẳng Bit, nén JPEG và phân tích ẩn miền biến đổi.

Phân tích ẩn biểu đồ phân tích biểu đồ của khung video bìa và khung video stego để phát hiện sự hiện diện của thông tin bí mật. Biểu đồ là biểu diễn đồ họa của các giá trị pixel của hình ảnh dựa trên phân phối. Khi một video bìa được điều khiển để nhúng thông tin bí mật, biểu đồ của video stego bị ảnh hưởng, Việc nhúng thông tin bí mật có thể không nhìn thấy được bằng mắt thường, nhưng khi biểu đồ này được vẽ và so sánh với video bìa gốc, ngay cả thao tác nhỏ nhất cũng có thể được phát hiện.

Kiểm tra Chi-square là một kỹ thuật phân tích dữ liệu phổ biến được sử dụng để phát hiện sự hiện diện của thông tin bí mật. Thử nghiệm này hoạt động bằng cách quan sát sự giống nhau giữa sự kiện trong thời gian thực và kết quả mong đợi. Nó sử dụng phân phối tần suất để xác định tính ngẫu nhiên trong video. Các giá trị thấp hơn của bài kiểm tra cho thấy mức độ ngẫu nhiên cao hơn, xác nhận sự hiện diện của thông điệp bí mật. Giá trị cao hơn có nghĩa là mức độ ngẫu nhiên thấp hơn và chứng tỏ không có sự can thiệp nào trong video.

RS Steganalysis là một công cụ mạnh mẽ khác được giới thiệu bởi Jiri Fridrich và cộng sự. Phân tích RS được sử dụng để phát hiện thông tin bí mật được nhúng bằng các phương pháp dựa trên LSB. Nó so sánh các giá trị pixel của hình ảnh trong miền không gian. Việc lựa chọn các cặp pixel khác nhau dựa trên phương pháp, đôi khi các pixel lân cận được chọn và những lần khác, các pixel từ các khối khác nhau được chọn để so sánh. Các nhóm pixel này được gọi là Nhóm số ít (S) và Nhóm thông thường (R). Sự hiện diện của bí mật được xác định bằng cách nhóm các pixel dựa trên phân phối tần số và phân tích LSB của stego và video bìa. Các LSB được lật và chọn ngẫu nhiên để phát hiện thông điệp bí mật. Phân tích RS có độ tin cậy tốt hơn so với kiểm định Chi-square.

\subsection{Phân tích ẩn dựa trên đặc trưng}
Các đặc trưng là một phần quan trọng của một hình ảnh. Các phương pháp phân tích dữ liệu dựa trên đặc trưng trích xuất các đặc trưng từ hình ảnh và phân tích các đặc trưng để phát hiện sự hiện diện của thông tin bí mật. Các đặc trưng này có thể được sử dụng thêm trong việc huấn luyện bộ phân loại để tự động phát hiện thông tin bí mật bằng thuật toán học máy [106]. Các phương pháp ghi mật mã chênh lệch giá trị pixel (PVD) che giấu nhiều bit thông tin bí mật hơn trong các vùng mượt mà của ảnh bìa hơn là trong các vùng phức tạp. Phân tích biểu đồ của kỹ thuật ghi mã PVD cho thấy phân phối Laplace. Một phương pháp phân tích ẩn dựa trên đặc trưng được sử dụng để phát hiện sự hiện diện của thông tin bí mật. Do PVD có phân phối Laplace, nên phân bố tần số dự kiến của hình ảnh thu được bằng cách sử dụng bất kỳ thử nghiệm ngẫu nhiên nào. Các giá trị mong đợi được so sánh với các giá trị quan sát được và mức độ tương tự được tính toán. Nếu độ tương tự nằm dưới một ngưỡng nhất định, thì hình ảnh không bị giả mạo, nếu không thì nó có một số thông tin được nhúng.